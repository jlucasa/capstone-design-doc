\subsection{System Architecture}

After an analysis of our system components and the different features they constitute, we devised the following rough requirements analysis diagram, subject to change as the system improves over time:

\begin{figure}[H]
    \centering
    \includesvg[scale=0.4]{assets/system-architecture-diagram.svg}
    \caption{System Architecture Diagram}
    \label{fig:system_architecture_diagram}
\end{figure}

In this system, the employee is given an account from their IT department (meaning there is no account creation process, as is reflected in the diagram), and they log in using our secure login system component.  This takes them to the application frontend, which is the hub of all activity on the application. This is sourced out to the following components:

\begin{itemize}[wide, labelindent=0pt]
    \item \textbf{Credential system.} This includes verifying usernames and passwords for accounts already given to us by the institution. Importantly, in production, this will not include registering new accounts. This will communicate with the login and application module.
    \item \textbf{The document parser.} This includes annotation, verification, and parsing of documents. The verifier reaches out to an API managed by the bank that we will communicate with. The verifier then communicates with the parser, passing in its results to it. The parser can then communicate with the annotator in the case of annotation, which communicates with the document persistence component.
    \item \textbf{Document persistence.} This constitutes the persistent, safe, and secure storage of documents in the system. It is in charge of sending and receiving documents for annotation and auditing by both the system and its given user. This will largely communicate with the Parser and Auditing Capabilities module.
    \item \textbf{Procedure management.} This will include storing rules that customize the way our customer can verify existing documents and allow them to create rules for how to process new document types in our system. Procedure management will be plugged into the frontend to allow users to enter a new interface where they can create and update rules.
    \item \textbf{Auditing Capabilities.} This will allow an internal auditor to go through previously verified documents and flag documents that were incorrectly verified the first time around and then allows the auditor to save these flags and new annotations they will create. This will communicate with the document persistence API to fetch documents that have been verified previously.
\end{itemize}