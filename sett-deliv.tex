\documentclass[11pt, fleqn]{article}

\usepackage[T1]{fontenc}
\usepackage[utf8]{inputenc}

\usepackage[margin=1in, footskip=.25in]{geometry}
\usepackage{amsmath,amsthm,amssymb,amsfonts}
\usepackage{color}
\usepackage{setspace}
\usepackage{enumitem}
\usepackage{fancyhdr}
\usepackage{graphicx}
\usepackage{float}
\usepackage{booktabs, makecell, caption, threeparttable, tabularx}
\usepackage{svg}
\usepackage[notmath]{sansmathfonts}
\usepackage{wrapfig}
\usepackage{cellspace}
\usepackage{xparse}
\usepackage{keyval}
\usepackage{listings}
\usepackage{caption}
\usepackage[xetex, bookmarks=true, breaklinks, pagebackref=true]{hyperref}
\usepackage{bookmark}
\usepackage{fontspec}
\usepackage{titlesec}

\restylefloat{figure}
\restylefloat{table}

% \renewcommand{\familydefault}{\sfdefault}
\setlist[enumerate]{topsep=0pt,itemsep=-1ex,partopsep=1ex,parsep=1ex}
% \renewcommand\tabularxcolumn[1]{m{#1}}% for vertical centering text in X column

\setlength{\cellspacetoplimit}{4pt}
\setlength{\cellspacebottomlimit}{4pt}

\hypersetup{
    colorlinks=true,
    linktoc=all,
    linkcolor=blue,
    filecolor=magenta,      
    urlcolor=cyan,
    pdftitle={Bank Document Verifier Design Document}
}

\urlstyle{same}

\usepackage[parfill]{parskip}

\ExplSyntaxOn

\seq_new:N \l_local_enum_seq

\NewDocumentCommand{\uctableenum}{~m} {
    \begin{enumerate}[wide, labelindent=0pt]
        \seq_set_split:Nnn \l_local_enum_seq {#1}
        \item 
        \seq_use:Nn \l_local_enum_seq {\item}
    \end{enumerate}
}

\NewDocumentCommand{\uctableitem}{~m}{
    \begin{itemize}[wide, labelindent=0pt]
        \seq_set_split:Nnn \l_local_enum_seq {#1}
        \item 
        \seq_use:Nn \l_local_enum_seq {\item}
    \end{itemize}
}

\ExplSyntaxOff

\newcommand{\setlststylejs}{
    \definecolor{lightgray}{rgb}{.9,.9,.9}
    \definecolor{darkgray}{rgb}{.4,.4,.4}
    \definecolor{purple}{rgb}{0.65, 0.12, 0.82}
    \definecolor{darkgreen}{RGB}{0, 100, 0}
    
    \lstdefinelanguage{JavaScript}{
      keywords={typeof, new, true, false, catch, function, return, null, catch, switch, var, if, in, while, do, else, case, break},
      keywordstyle=\color{blue}\bfseries,
      ndkeywords={class, export, boolean, throw, implements, import, this},
      ndkeywordstyle=\color{darkgray}\bfseries,
      identifierstyle=\color{black},
      sensitive=false,
      comment=[l]{//},
      morecomment=[s]{/*}{*/},
      commentstyle=\color{darkgreen}\ttfamily,
      stringstyle=\color{red}\ttfamily,
      morestring=[b]',
      morestring=[b]"
    }
    
    \lstset{
       language=JavaScript,
       backgroundcolor=\color{lightgray},
       extendedchars=true,
       basicstyle=\footnotesize\ttfamily,
       showstringspaces=false,
       showspaces=false,
       numberstyle=\footnotesize,
       numbersep=9pt,
       tabsize=2,
       breaklines=true,
       showtabs=false,
       captionpos=b
    }
}

\pagestyle{fancy}
\renewcommand{\headrulewidth}{0pt}
\setlength{\headheight}{53.51314pt}
\lhead{\textbf{Bank Document Verifier} \\ Design Document}
\rhead{\includesvg[width=1.25cm]{assets/logo.svg}}
\cfoot{\thepage}

\titleformat*{\section}{\LARGE\bfseries}
\titleformat*{\subsection}{\Large\bfseries}
\titleformat*{\subsubsection}{\large\bfseries}
\titleformat*{\paragraph}{\large\bfseries}
\titleformat*{\subparagraph}{\large\bfseries}

\setlststylejs

\begin{document}

\begin{titlepage}

\newcommand{\HRule}{\rule{\linewidth}{0.5mm}} % Defines a new command for the horizontal lines, change thickness here

\center % Center everything on the page
 
%----------------------------------------------------------------------------------------
%	HEADING SECTIONS
%----------------------------------------------------------------------------------------

\textsc{\LARGE University of Utah}\\[0.5cm] % Name of your university/college

\begin{figure}[h]
    \centering
    \includesvg[scale=0.75]{assets/u-logo.svg}
    \label{fig:u_logo}
\end{figure}

\vspace{0.5cm}
\textsc{\Large Senior Capstone Design}\\[0.5cm] % Major heading such as course name
\textsc{\large CS 4000}\\[0.5cm] % Minor heading such as course title



%----------------------------------------------------------------------------------------
%	TITLE SECTION
%----------------------------------------------------------------------------------------

\HRule \\[0.5cm]
{\LARGE \textbf{Bank Document Verifier}}\\[0.25cm] % Title of your document
{\large \textsc{Design Document}}\\[0.25cm]

\begin{figure}[h]
    \centering
    \hspace{8mm}
    \includesvg[scale=0.4]{assets/logo.svg}
    \label{fig:logo}
\end{figure}
\HRule \\[0.5cm]

%----------------------------------------------------------------------------------------
%	AUTHOR SECTION
%----------------------------------------------------------------------------------------

% \begin{minipage}{0.4\textwidth}
% \begin{center} \large
% \emph{Authors:} \\
% Jared \textsc{Amen} \\
% Jaecee \textsc{Naylor} \\
% Trevor \textsc{Dalton} \\
% \end{center}

% \end{minipage}\\[2cm]

% If you don't want a supervisor, uncomment the two lines below and remove the section above
\Large \emph{Authors:}\\
Jared \textsc{Amen} \\ % Your name
Jaecee \textsc{Naylor} \\
Trevor \textsc{Dalton} \\[2cm]

%----------------------------------------------------------------------------------------
%	DATE SECTION
%----------------------------------------------------------------------------------------

{\large \today}\\[2cm] % Date, change the \today to a set date if you want to be precise

\vfill % Fill the rest of the page with whitespace

\end{titlepage}


% \end{document}

\newpage

\section{Software Engineering Tools \& Techniques}

\subsection{Software Development Methodology}

After much discussion, our team has elected to move forward with a \textbf{SCRUM software development method with two-week sprints}. To help enforce the precedent that all team members should produce some amount of considerable work week-to-week (to fit in line with team deliverables and check-ups), we will hold midpoint evaluations at the end of the first week of a given sprint (discussed in the \href{sec:team_comm_meeting}{section on team communications and meeting schedules}).

\subsection{Tools}

The following considerations have been made regarding tools used in our system. As development progresses, we expect to add more tools to these list as necessary.

\medskip

\begin{itemize}[nosep]
    \item \textbf{IDE:} Visual Studio Code will be our primary IDE of use.
    \item \textbf{Main Library:} We have decided to use Azure for DBMS and various cloud services.
    \item \textbf{Testing Plans and Tools:} Our SCRUM sprints will be designed around testing, with the last couple days before a sprint completes being dedicated to testing. Aside from that, we might look into using tools such as Selenium for simulating typical usage of a browser to detect bugs. Additionally, test-driven development has been discussed as an option -- while more research would need to be done on the matter, it is conceivable that this would be an effective strategy for us to implement.
    \item \textbf{Version Control:} We will continue to use the university's distribution of Gitlab for both internal and production builds of our code.
    \item \textbf{Issue Tracking:} Again, as above, the university's distribution of Gitlab will work well for us here. Additionally, we will plan out issues as stories with attached milestones, epics, weights, tags, and comment systems to easily track issues and their context.
\end{itemize}

\subsection{Code Reviews}

After much discussion, we have elected to implement the following system, considered on a case-by-case basis for issues, with all commits put into pull requests:

\medskip

\begin{itemize}[nosep]
    \item \textbf{Minor to moderate non-code changes, minor code changes:} No code reviews necessary.
    \item \textbf{Moderate code changes:} Code review done by one other team member.
    \item \textbf{Major code changes:} Code reviews done by both other team members.
\end{itemize}

\subsection{Documentation}

When devising a plan for documentation, we wanted the process of porting code comments to readable, easily accessible documentation to be as simple as possible. As such, we are looking into using tools such as \href{https://jsdoc.app/index.html}{JSDoc}, \href{https://rfsber.home.xs4all.nl/Robo/}{ROBODoc} and \href{https://docs.python.org/3/library/pydoc.html}{pydoc} for automatic documentation generation from in-code documentation.

For code styling, we have decided that function/variable declarations will be written in \verb|camelCase|, and all other styling decisions will be made according to Google's officially posted \href{https://google.github.io/styleguide/}{style guides}.

\medskip

Additionally, we are looking to implement a header comment standard for all code files generated -- it will be as such:

\lstinputlisting[caption=Header comment standard for all code files generated]{assets/code/header-comment-example.js}

Finally, we want to implement a commenting standard for all code files:

\medskip

\begin{itemize}[nosep]
    \item \textbf{Comment before code creation:} Header description
    \item \textbf{Comment during code creation:} Function descriptions (including parameters)
    \item \textbf{Comment after/discretionary for complicated code:} Block descriptions
\end{itemize}

\subsection{Team Communication \& Meeting Schedule}
\label{sec:team_comm_meeting}

\textbf{All team communication will take place on our Discord server.}

\medskip

For team meetings, in accordance with our decision to work with two-week sprints which have strict midpoint evaluations halfway through, we have reached the following schedule:

\medskip

\begin{itemize}[nosep]
    \item \textbf{Sprint planning meeting:} a 30-45 minute sprint planning meeting at the beginning of every sprint (Monday)
    \item \textbf{Discord text standups:} updates posted to a \verb|#wednesday-updates| Discord text channel every Wednesday at 6pm.
    \item \textbf{Sprint midpoint evaluations:} a 30 minute update on everyone's work, roughly halfway through any given sprint.
\end{itemize}

\medskip

Finally, we have agreed that any member of the team that is currently working on the project should be in our Discord server's voice room.

\end{document}